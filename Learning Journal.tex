\documentclass[12pt]{article}
 
 %-Packages
\usepackage[margin=1in]{geometry} 
\usepackage{titlesec}
\usepackage{hyperref}

%-Counters
\newcounter{problem} \setcounter{problem}{1}
%\addtocounter{section}{1}


%-Commands
\newcommand{\actionlog}[1]{\paragraph{Action Log ID: \theproblem\ -}{#1} ~\\ \addtocounter{problem}{1}}
\newcommand{\newweek}[1]{\addtocounter{section}{1} \underline{\section{Assignments - Week \thesection}}\\[5mm]}
\newcommand{\intention}[1]{\textbf{Intention:}{\textnormal\ #1} \newline}
\newcommand{\action}[1]{\textbf{Action:}{\textnormal\ #1} \newline}
\newcommand{\result}[1]{\textbf{Result:}{\textnormal\ #1} \newline}
\newcommand{\solution}[1]{\textbf{Solution:}{\textnormal\ #1} \newline}
\newcommand{\learning}[2]{\item \textit{#1} \textnormal{#2}}
\newcommand{\remember}{\paragraph{Key Points to Remember:}}
\newcommand{\question}[1]{\paragraph{Question: {\textnormal{\textit{#1}}} ~\\}}

%-Config
\hypersetup{
    colorlinks,
    citecolor=black,
    filecolor=black,
    linkcolor=navy,
    urlcolor=black
}

\title{Learning Journal}
\author{Aaron Hammond\\
FOAR705 - Digital Humanities}
\date{Semester 2, 2019}

%---------------------------------------------------------------
\begin{document}
%---------------------------------------------------------------

\begin{titlepage}
\maketitle
\def\contentsname{\empty}
\tableofcontents
\end{titlepage}

%---------------------------------------------------------------Week 2

\newweek{}


\subsection{Formatting data tables in Spreadsheets}
\linebreak

\textit{1.What are some common challenges with formatting data in spreadsheets and how can we avoid them?}\\
\\
The largest challenge with spreadsheets is formatting the information in a manner in which the information can be easily read regardless of who looks at it. The spreadsheet should be written that no prior knowledge is required to decode the information inside it. This means restricting the use of colours or any sort of coding device and instead extend the data table to include the additional information.\\
\\
Exercises:\\
\textit{1. Identify what is wrong with this spreadsheet (SAFI messy). Discuss the steps you would need to take to clean up the two tabs, and to put them all together in one spreadsheet}\\
\begin{enumerate}
    \item Rename SAFI\_messy to SAFI\_raw
    \item Create a new spreadsheet titled SAFI\_0.1
    \item In a single spreadsheet copy and paste the information from Mozambique 
    \item Label the headings of the columns as follows A. country B. key\_id C. type\_roof D. type\_floor E. no\_rooms F. includes\_barn G. oxen H. poultry I. goats J. cows K. total\_livestock L. no\_plots M. water\_use N. comments (for cowshed, look after cows, only in summer, dead cows info)
    \item Organise the data according to the information as possible, consolidating as many lines as needed and using Cut+Copy functions to ensure no doubling of data or missing of information
    \item Copy data from Tanzania into the same spreadsheet (delete empty tab)
    \item Populate the country field while organise the data according to the same principles used for Mozambique
    \item Delete any obviously incorrect data (e.g. -999) and replace with (Null) value to indicate the information collected for this field is not entirely credible
    \item Save changes
\end{enumerate}
\textit{2. Discuss this data with a partner and make a list of some of the types of metadata that should be recorded about this dataset}\\
\begin{enumerate}
    \item Who attended these interviews?
    \item What does the use of liv mean in the columns?
    \item How often is \textit{frequently ?}
    \item Does months\_lack\_food mean no meals or not a sufficient number of meals?
    \item instanceID is used in what type of software?
\end{enumerate}

\subsubsection{Find examples of problem in data produced by your discipline}
Within my discipline of Anthropology, locating problem data is very difficult. Typically ethnographies are written using field notes that are supported by academic literature. In saying this it is very difficult to locate problem sets of data as the public typically and privy to them.

\subsection{Learning Journal \date{15/08/2019}} 
\vspace{5mm}

\actionlog{No Timestamp}
\begin{itemize}
    \item Intention: Create a line break in LaTeX
    \item Action: Used the Rich Text tab and pressed enter
    \item Result: No line break
    \item Improvements or Solutions: Researched and found command "newline" to assist with formatting.
\end{itemize}

\actionlog{No Timestamp}
\begin{itemize}
    \item Intention: Create a custom title for each page
    \item Action: Used the 'title' command and 'maketitle' in different combinations
    \item Result: No title added
    \item Improvements or Solutions: Instead use the title page function for the initial title and utilise 'sections' instead to group areas of my work.
\end{itemize}

\actionlog{No Timestamp}
\begin{itemize}
    \item Intention: Create a numbered bullet list
    \item Action: Used 'itemize' command
    \item Result: Bulletted list without numbers
    \item Improvements or Solutions: Instead utilise the enumerate command for numbered lists
\end{itemize}

\actionlog{No Timestamp}
\begin{itemize}
    \item Intention: Use the underscore in naming conventions
    \item Action: Quoted a filename which utilised the \_ character
    \item Result: Massive issues in formatting and text running off the page
    \item Improvements or Solutions: When using underscores the backslash command must be used before the \_ symbol to ensure formatting remains intact
\end{itemize}

\actionlog{No Timestamp}
\begin{itemize}
    \item Intention: Create a counter so i don't have to manually count my problems
    \item Action: Created a counter and attempted to use stepcounter to print and increase it's value by 1 
    \item Result: Step counter did not print any characters
    \item Improvements or Solutions: Instead i used the counter command and referenced it using 'thecounter' and increased its value by 1 using 'addtocounter'afterwards
\end{itemize}

\actionlog{No Timestamp}
\begin{itemize}
    \item Intention: Fix the formatting issues caused by ending lists with vspace
    \item Action: Added vspace to the first Problem\_ID
    \item Result: All Problem\_IDs after the first are aligned differently
    \item Improvements or Solutions: Perhaps using a command other than vspace for formatting
\end{itemize}

\clearpage

%------------------------------------------------------Week 3

\newweek{}

\subsection{Data Carpentry (Spreadsheets)}
\subsubsection{Date formats in spreadsheets}
\paragraph{Question: What year is shown in the year column?} ~\\
When adding data point '17/11' into the interview date column the Year column (formatted as =YEAR(A\$)) populates the field with '2019' (or the current year)
\newline
\paragraph{Key Points to Remember}
\begin{itemize}
    \item When working with dates within spreadsheets it is best to divide the three points of data into their own columns
    \item Do not allow the software to store the information as dates but integers instead to ensure they can be translated correctly to other software
    \item Excel will store dates as the current year unless specified otherwise
\end{itemize}

\subsubsection{Quality Assurance}

\paragraph{Key Points to Remember}
\begin{itemize}
    \item Data Validation tools help to mitigate errors and provide restrictions on the type of data that can be placed into columns
    \item Error messages can be customized as warnings or hard stops.
    \item Drop-down lists can be created per each data validation rule or can be sourced from within the document
    \item Overuse of these could cause issues if encountering unexpected  conditions or unique circumstances 
\end{itemize}

\subsubsection{Exporting Data}
\paragraph{Key Points to Remember}
\begin{itemize}
    \item File formats such as .xls and .xlsx are proprietary format and risk not being supported by future technology
    \item To ensure a document is more timeless and usable within a range of software, data should be saved as a .csv (comma seperated values) file type
    \item Within csv files ensure that any commas used in data fields are written within double-quotation marks to ensure that data is not reformatted incorrectly on re-opening
    \item Because it is formatted simply with commas and lines, .csv files can be edited in any text edit (such as notepad)
\end{itemize}

\subsection{Learning Journal}

\subsubsection{New Techniques and Commands Learned:}
\begin{itemize}
    \learning{subsubsection}{can be used to keep effective formatting within a subsection}
    \learning{tilde+doublebackslash}{can be used to force a line break after the paragraph command}
    \learning{verb}{can be used to ignore coding commands}
    \learning{newcommand[]}{can be used to create macros and within brackets you specify how many arguements you would like to include, arguments are referenced with the 'pound sign + number' combination}
    \learning{setcounter}{is used to set the value of a counter by specifying the counter name and the value within two separate curly bracket sets}
    \learning{backslash+space}{can be used to force a space}
    \learning{the+countername}{is the command to print the counter's value}
\end{itemize}

\subsubsection{Action Log}

\actionlog{22/08/2019 15:21}
\intention{Neatly format within a Subsection}
\action{attempted to use paragraph command to a standardized formatting}
\result{new line did not include referencing numbers for location}
\solution{Use the subsubsection command instead}

\actionlog{22/08/2019 15:36}
\intention{Write the cell variable from excel into Overleaf using the \$ symbol}
\action{Write the symbol as i would have any other}
\result{remaining code was highlighted in red indicating there was something missing or the code wouldn't recompile correctly}
\solution{Use the backslash character before the \$ to ensure it prints correctly}

\actionlog{22/08/2019 15:43}
\intention{Use the paragraph command to introduce and emphasize text then continue with normal text underneath}
\action{Attempted the paragraph command followed by \textit{newline} command however no return was created}
\solution{I utilized the \textit{tilde+double-backslash} command to force a new line}

\actionlog{22/08/2019 15:53}
\intention{Create a new command that automatically formatted the 'Things that i learned' area - having the command and it's action using differently formatted text}
\result{All the text came through \textit{italic}}
\solution{When creating a new command ensure to select the correct number of arguments in the first bracket}

\actionlog{22/08/2019 16:32}
\intention{Print a counter's value by referencing its name}
\action{Within the \textit{newcommand} for action logs I wrote the counter's name to be referenced}
\result{Counter was not printed}
\solution{In order to print a counter it must be preceded by \textit{the} e.g. If your counter is named dinosaur it mus be printed by typing \textit{thedinosaur}}

\actionlog{22/08/2019 16:56}
\intention{Create a new counter that inherently had a value of 1}
\action{Used the \textit{newcounter} command and placed the value that I wanted in the following bracket}
\result{Counter didn't appear when I attempted to print it}
\solution{The bracket isn't to dictate a value for the counter but what section it corresponds to, a second command of \textit{setcounter} must be used to explicitly set it}

\actionlog{22/08/2019 17:11}
\intention{Simply add a space after inserting a printed counter}
\action{Added an extra space after printing it with 'theproblem'}
\result{No space was added}
\solution{In order to ensure an additional space is added i used the \textit{backslash+space} command to force it there}

\actionlog{22/08/2019 17:41}
\intention{Print what is specifically written in the code}
\action{I placed the text that i wanted printed within quotation marks and insert brackets}
\result{It created bizzarre characters and was not at all what i wanted}
\solution{I can use the \textit{verb} and utilize | marks to print without translating the commands}
\newpage

%---------------------------------------------------------Week4
\newweek{}
\subsection{Data Carpentry (Shell)}
\subsubsection{Introducing the Shell}
%Shell1
\question{What is a command shell and why would I use one?}
A shell is a text-based command line interface that responds explicitly to commands entered. This type of interface is useful when conducting highly-repetitive tasks and as a terminal to send commands to a remote machine.  
\remember
\begin{itemize}
    \item Shell's main advantage is to increase the action-to-keystroke ratio by automating repetitive tasks
    \item \textit{ls} command can be used to list files within the current directory
    \item \textit{cd} command can be used to change directory
\end{itemize}
%Shell2
\subsubsection{Navigating Files and Directories}

\question{Starting from /Users/amanda/data, which of the following commands could Amanda use to navigate to her home directory, which is /Users/amanda?}
Amanda would need to use the \textit{cd \~} command

\question{Using the filesystem diagram below, if pwd displays /Users/thing, what will ls -F ../backup display?}
4. original/ pnas\_final/ pnas\_sub/

\question{Using the filesystem diagram below, if pwd displays /Users/backup, and -r tells ls to display things in reverse order, what command(s) will result in the following output}


\question{How can I move around on my computer?}
The command \textit{cd} is used to move around the computer.

\question{Can I see what files and directories I have?}
The command \textit{ls} will list the files and filders within a the specified folder. If no folder is specified the currently occupied folder items will be displayed.

\question{How can I specify the location of a file or directory on my computer?}
This can be achieved by using the -F option and dictating the folder location e.g. \textit{ls -f users/aaron/documents}

\remember
\begin{itemize}
    \item \textit{pwd} command can be used to print the current working directory
    \item \textit{ls --help} command will list all the options for the ls command
    \item \textit{cd \~, .. or -} Tilde goes home, .. goes up and - goes back
    
\end{itemize}

\subsubsection{Working With Files and Directories}

\question{How can I create, copy, and delete files and directories?}
Command \textit{mkdir} can be used to create a folder \\
Command \textit{touch <filename>} can be used to create a blank file \\
Command \textit{mv} will move desired file(s) to specified location
Command \textit{cp} will copy desired files to specified location (if same location it will be copied and renamed)
Command \textit{rm} will delete a file, use -r option to delete a directory and -i for a prompt

\question{When might you want to create a file this way? (using touch command}
The file size is 0kb and it could be used when running loop commands or to quickly store data in a file using a script.

\question{Fill in the blanks to move these files to the current folder (i.e., the one she is currently in):}
\$ mv ../analyzed/sucrose.dat ../analyzed/maltose.dat .

\question{After creating and saving this file you realize you misspelled the filename! You want to correct the mistake, which of the following commands could you use to do so?}
mv statstics.txt statistics.txt

\question{What is the output of the closing ls command in the sequence shown below?}
2.recombine

\question{What happens when we execute rm -i thesis\_backup/quotations.txt? Why would we want this protection when using rm?}
This will provide a prompt forcing you to enter another command to delete the file.

\question{In the example below, what does cp do when given several filenames and a directory name?}
This will copy the two files into the destination folder (backup)

\question{In the example below, what does cp do when given three or more file names?}
Command cannot be executed as the file is not a location.

\question{When run in the molecules directory, which ls command(s) will produce this output?}
ls *t??ne.pdb

\question{The fructose.dat and sucrose.dat files contain output from her data analysis. What command(s) does she need to run so that the commands below will produce the output shown?}
Question already states that fructose.dat and sucrose.dat are in the analyzed folder. But if you would like to move them elsewhere you would use mv *.dat followed by your location

\question{Which of the following set of commands would achieve this objective? What would the other commands do?}
$ mkdir 2016-05-20\\
$ mkdir 2016-05-20/data\\
$ mkdir 2016-05-20/data/processed\\
$ mkdir 2016-05-20/data/raw\\
\\
Would create the desired folder structure.

\question{How can I edit files?}
Text files can be edited by using the nano command or renamed using the \textit{mv} command

\subsection{Learning Journal}
\subsubsection{New Techniques and Commands Learned:}
\begin{itemize}
    \learning{0mm}{Can be used to apply vspace without the command}
    \learning{rule}{Command can be used to rule a line accross the page}
    \learning{empty}{Value can be used to empty a default field (e.g. contentsname}
    \learning{asteriks}{At the end of a command (such as section) can be used to prevent it from including a number}
    \learning{hypperref}{Package can be used to link the table of contents to its location within the document}
\end{itemize}

\subsubsection{Action Log}

\actionlog{29/08/2019 14:27}
\intention{Place a border around the title page in elaboration I}
\action{Attempted to use the \textit{frame} command around title page}
\result{Everything was removed except the data within a small frame}
\solution{Found }

%Shell3
\end{document}