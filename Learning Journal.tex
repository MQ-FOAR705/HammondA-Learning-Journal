\documentclass[12pt]{article}
 
 %-Packages
\usepackage[margin=1in]{geometry} 
\usepackage{titlesec}
\usepackage{hyperref}
\usepackage{xcolor}
\usepackage{lmodern}

%-Counters
\newcounter{problem} \setcounter{problem}{1}
\addtocounter{section}{1}


%-Commands
\newcommand{\actionlog}[1]{\paragraph{Action Log ID: \theproblem\ -}{#1} ~\\ \addtocounter{problem}{1}}
\newcommand{\newweek}[1]{\underline{\section{Assignments - Week \thesection}}\\[5mm]}
\newcommand{\intention}[1]{\textbf{Intention:}{\textnormal\ #1} \newline}
\newcommand{\action}[1]{\textbf{Action:}{\textnormal\ #1} \newline}
\newcommand{\result}[1]{\textbf{Result:}{\textnormal\ #1} \newline}
\newcommand{\solution}[1]{\textbf{Solution:}{\textnormal\ #1} \newline}
\newcommand{\learning}[2]{\item \textit{#1} \textnormal{#2}}
\newcommand{\remember}{\paragraph{Key Points to Remember:}}
\newcommand{\question}[1]{\paragraph{Question: {\textnormal{\textit{#1}}} ~\\}}

%-Config

\definecolor{hyperblue}{RGB}{67, 69, 137}
\hypersetup{
    colorlinks,
    citecolor=black,
    filecolor=black,
    linkcolor=hyperblue,
    urlcolor=black,
    linktoc=all
}

\title{Learning Journal}
\author{Aaron Hammond\\
FOAR705 - Digital Humanities}
\date{Semester 2, 2019}

%---------------------------------------------------------------
\begin{document}
%---------------------------------------------------------------

\begin{titlepage}
\maketitle
\vspace{-2.5cm}
\def\contentsname{\empty}
\tableofcontents
\end{titlepage}

%---------------------------------------------------------------Week 2

\newweek{}


\subsection{Formatting data tables in Spreadsheets}
\linebreak

\textit{1.What are some common challenges with formatting data in spreadsheets and how can we avoid them?}\\
\\
The largest challenge with spreadsheets is formatting the information in a manner in which the information can be easily read regardless of who looks at it. The spreadsheet should be written that no prior knowledge is required to decode the information inside it. This means restricting the use of colours or any sort of coding device and instead extend the data table to include the additional information.\\
\\
Exercises:\\
\textit{1. Identify what is wrong with this spreadsheet (SAFI messy). Discuss the steps you would need to take to clean up the two tabs, and to put them all together in one spreadsheet}\\
\begin{enumerate}
    \item Rename SAFI\_messy to SAFI\_raw
    \item Create a new spreadsheet titled SAFI\_0.1
    \item In a single spreadsheet copy and paste the information from Mozambique 
    \item Label the headings of the columns as follows A. country B. key\_id C. type\_roof D. type\_floor E. no\_rooms F. includes\_barn G. oxen H. poultry I. goats J. cows K. total\_livestock L. no\_plots M. water\_use N. comments (for cowshed, look after cows, only in summer, dead cows info)
    \item Organise the data according to the information as possible, consolidating as many lines as needed and using Cut+Copy functions to ensure no doubling of data or missing of information
    \item Copy data from Tanzania into the same spreadsheet (delete empty tab)
    \item Populate the country field while organise the data according to the same principles used for Mozambique
    \item Delete any obviously incorrect data (e.g. -999) and replace with (Null) value to indicate the information collected for this field is not entirely credible
    \item Save changes
\end{enumerate}
\textit{2. Discuss this data with a partner and make a list of some of the types of metadata that should be recorded about this dataset}\\
\begin{enumerate}
    \item Who attended these interviews?
    \item What does the use of liv mean in the columns?
    \item How often is \textit{frequently ?}
    \item Does months\_lack\_food mean no meals or not a sufficient number of meals?
    \item instanceID is used in what type of software?
\end{enumerate}

\subsubsection{Find examples of problem in data produced by your discipline}
Within my discipline of Anthropology, locating problem data is very difficult. Typically ethnographies are written using field notes that are supported by academic literature. In saying this it is very difficult to locate problem sets of data as the public typically and privy to them.

\subsection{Learning Journal \date{15/08/2019}} 
\vspace{5mm}

\actionlog{No Timestamp}
\begin{itemize}
    \item Intention: Create a line break in LaTeX
    \item Action: Used the Rich Text tab and pressed enter
    \item Result: No line break
    \item Improvements or Solutions: Researched and found command "newline" to assist with formatting.
\end{itemize}

\actionlog{No Timestamp}
\begin{itemize}
    \item Intention: Create a custom title for each page
    \item Action: Used the 'title' command and 'maketitle' in different combinations
    \item Result: No title added
    \item Improvements or Solutions: Instead use the title page function for the initial title and utilise 'sections' instead to group areas of my work.
\end{itemize}

\actionlog{No Timestamp}
\begin{itemize}
    \item Intention: Create a numbered bullet list
    \item Action: Used 'itemize' command
    \item Result: Bulletted list without numbers
    \item Improvements or Solutions: Instead utilise the enumerate command for numbered lists
\end{itemize}

\actionlog{No Timestamp}
\begin{itemize}
    \item Intention: Use the underscore in naming conventions
    \item Action: Quoted a filename which utilised the \_ character
    \item Result: Massive issues in formatting and text running off the page
    \item Improvements or Solutions: When using underscores the backslash command must be used before the \_ symbol to ensure formatting remains intact
\end{itemize}

\actionlog{No Timestamp}
\begin{itemize}
    \item Intention: Create a counter so i don't have to manually count my problems
    \item Action: Created a counter and attempted to use stepcounter to print and increase it's value by 1 
    \item Result: Step counter did not print any characters
    \item Improvements or Solutions: Instead i used the counter command and referenced it using 'thecounter' and increased its value by 1 using 'addtocounter'afterwards
\end{itemize}

\actionlog{No Timestamp}
\begin{itemize}
    \item Intention: Fix the formatting issues caused by ending lists with vspace
    \item Action: Added vspace to the first Problem\_ID
    \item Result: All Problem\_IDs after the first are aligned differently
    \item Improvements or Solutions: Perhaps using a command other than vspace for formatting
\end{itemize}

\clearpage

%------------------------------------------------------Week 3

\newweek{}

\subsection{Data Carpentry (Spreadsheets)}
\subsubsection{Date formats in spreadsheets}
\paragraph{Question: What year is shown in the year column?} ~\\
When adding data point '17/11' into the interview date column the Year column (formatted as =YEAR(A\$)) populates the field with '2019' (or the current year)
\newline
\paragraph{Key Points to Remember}
\begin{itemize}
    \item When working with dates within spreadsheets it is best to divide the three points of data into their own columns
    \item Do not allow the software to store the information as dates but integers instead to ensure they can be translated correctly to other software
    \item Excel will store dates as the current year unless specified otherwise
\end{itemize}

\subsubsection{Quality Assurance}

\paragraph{Key Points to Remember}
\begin{itemize}
    \item Data Validation tools help to mitigate errors and provide restrictions on the type of data that can be placed into columns
    \item Error messages can be customized as warnings or hard stops.
    \item Drop-down lists can be created per each data validation rule or can be sourced from within the document
    \item Overuse of these could cause issues if encountering unexpected  conditions or unique circumstances 
\end{itemize}

\subsubsection{Exporting Data}
\paragraph{Key Points to Remember}
\begin{itemize}
    \item File formats such as .xls and .xlsx are proprietary format and risk not being supported by future technology
    \item To ensure a document is more timeless and usable within a range of software, data should be saved as a .csv (comma seperated values) file type
    \item Within csv files ensure that any commas used in data fields are written within double-quotation marks to ensure that data is not reformatted incorrectly on re-opening
    \item Because it is formatted simply with commas and lines, .csv files can be edited in any text edit (such as notepad)
\end{itemize}

\subsection{Learning Journal}

\subsubsection{New Techniques and Commands Learned:}
\begin{itemize}
    \learning{subsubsection}{can be used to keep effective formatting within a subsection}
    \learning{tilde+doublebackslash}{can be used to force a line break after the paragraph command}
    \learning{verb}{can be used to ignore coding commands}
    \learning{newcommand[]}{can be used to create macros and within brackets you specify how many arguements you would like to include, arguments are referenced with the 'pound sign + number' combination}
    \learning{setcounter}{is used to set the value of a counter by specifying the counter name and the value within two separate curly bracket sets}
    \learning{backslash+space}{can be used to force a space}
    \learning{the+countername}{is the command to print the counter's value}
\end{itemize}

\subsubsection{Action Log}

\actionlog{22/08/2019 15:21}
\intention{Neatly format within a Subsection}
\action{attempted to use paragraph command to a standardized formatting}
\result{new line did not include referencing numbers for location}
\solution{Use the subsubsection command instead}

\actionlog{22/08/2019 15:36}
\intention{Write the cell variable from excel into Overleaf using the \$ symbol}
\action{Write the symbol as i would have any other}
\result{remaining code was highlighted in red indicating there was something missing or the code wouldn't recompile correctly}
\solution{Use the backslash character before the \$ to ensure it prints correctly}

\actionlog{22/08/2019 15:43}
\intention{Use the paragraph command to introduce and emphasize text then continue with normal text underneath}
\action{Attempted the paragraph command followed by \textit{newline} command however no return was created}
\solution{I utilized the \textit{tilde+double-backslash} command to force a new line}

\actionlog{22/08/2019 15:53}
\intention{Create a new command that automatically formatted the 'Things that i learned' area - having the command and it's action using differently formatted text}
\result{All the text came through \textit{italic}}
\solution{When creating a new command ensure to select the correct number of arguments in the first bracket}

\actionlog{22/08/2019 16:32}
\intention{Print a counter's value by referencing its name}
\action{Within the \textit{newcommand} for action logs I wrote the counter's name to be referenced}
\result{Counter was not printed}
\solution{In order to print a counter it must be preceded by \textit{the} e.g. If your counter is named dinosaur it mus be printed by typing \textit{thedinosaur}}

\actionlog{22/08/2019 16:56}
\intention{Create a new counter that inherently had a value of 1}
\action{Used the \textit{newcounter} command and placed the value that I wanted in the following bracket}
\result{Counter didn't appear when I attempted to print it}
\solution{The bracket isn't to dictate a value for the counter but what section it corresponds to, a second command of \textit{setcounter} must be used to explicitly set it}

\actionlog{22/08/2019 17:11}
\intention{Simply add a space after inserting a printed counter}
\action{Added an extra space after printing it with 'theproblem'}
\result{No space was added}
\solution{In order to ensure an additional space is added i used the \textit{backslash+space} command to force it there}

\actionlog{22/08/2019 17:41}
\intention{Print what is specifically written in the code}
\action{I placed the text that i wanted printed within quotation marks and insert brackets}
\result{It created bizzarre characters and was not at all what i wanted}
\solution{I can use the \textit{verb} and utilize | marks to print without translating the commands}
\newpage

%---------------------------------------------------------Week4
\newweek{}
\subsection{Data Carpentry (Shell)}
\subsubsection{Introducing the Shell}
%Shell1
\question{What is a command shell and why would I use one?}
A shell is a text-based command line interface that responds explicitly to commands entered. This type of interface is useful when conducting highly-repetitive tasks and as a terminal to send commands to a remote machine.  
\remember
\begin{itemize}
    \item Shell's main advantage is to increase the action-to-keystroke ratio by automating repetitive tasks
    \item \textit{ls} command can be used to list files within the current directory
    \item \textit{cd} command can be used to change directory
\end{itemize}
%Shell2
\subsubsection{Navigating Files and Directories}

\question{Starting from /Users/amanda/data, which of the following commands could Amanda use to navigate to her home directory, which is /Users/amanda?}
Amanda would need to use the \textit{cd \~} command

\question{Using the filesystem diagram below, if pwd displays /Users/thing, what will ls -F ../backup display?}
4. original/ pnas\_final/ pnas\_sub/

\question{Using the filesystem diagram below, if pwd displays /Users/backup, and -r tells ls to display things in reverse order, what command(s) will result in the following output}


\question{How can I move around on my computer?}
The command \textit{cd} is used to move around the computer.

\question{Can I see what files and directories I have?}
The command \textit{ls} will list the files and filders within a the specified folder. If no folder is specified the currently occupied folder items will be displayed.

\question{How can I specify the location of a file or directory on my computer?}
This can be achieved by using the -F option and dictating the folder location e.g. \textit{ls -f users/aaron/documents}

\remember
\begin{itemize}
    \item \textit{pwd} command can be used to print the current working directory
    \item \textit{ls --help} command will list all the options for the ls command
    \item \textit{cd \~, .. or -} Tilde goes home, .. goes up and - goes back
    
\end{itemize}

\subsubsection{Working With Files and Directories}

\question{How can I create, copy, and delete files and directories?}
Command \textit{mkdir} can be used to create a folder \\
Command \textit{touch <filename>} can be used to create a blank file \\
Command \textit{mv} will move desired file(s) to specified location
Command \textit{cp} will copy desired files to specified location (if same location it will be copied and renamed)
Command \textit{rm} will delete a file, use -r option to delete a directory and -i for a prompt

\question{When might you want to create a file this way? (using touch command}
The file size is 0kb and it could be used when running loop commands or to quickly store data in a file using a script.

\question{Fill in the blanks to move these files to the current folder (i.e., the one she is currently in):}
\$ mv ../analyzed/sucrose.dat ../analyzed/maltose.dat .

\question{After creating and saving this file you realize you misspelled the filename! You want to correct the mistake, which of the following commands could you use to do so?}
mv statstics.txt statistics.txt

\question{What is the output of the closing ls command in the sequence shown below?}
2.recombine

\question{What happens when we execute rm -i thesis\_backup/quotations.txt? Why would we want this protection when using rm?}
This will provide a prompt forcing you to enter another command to delete the file.

\question{In the example below, what does cp do when given several filenames and a directory name?}
This will copy the two files into the destination folder (backup)

\question{In the example below, what does cp do when given three or more file names?}
Command cannot be executed as the file is not a location.

\question{When run in the molecules directory, which ls command(s) will produce this output?}
ls *t??ne.pdb

\question{The fructose.dat and sucrose.dat files contain output from her data analysis. What command(s) does she need to run so that the commands below will produce the output shown?}
Question already states that fructose.dat and sucrose.dat are in the analyzed folder. But if you would like to move them elsewhere you would use mv *.dat followed by your location

\question{Which of the following set of commands would achieve this objective? What would the other commands do?}
$ mkdir 2016-05-20\\
$ mkdir 2016-05-20/data\\
$ mkdir 2016-05-20/data/processed\\
$ mkdir 2016-05-20/data/raw\\
\\
Would create the desired folder structure.

\question{How can I edit files?}
Text files can be edited by using the nano command or renamed using the \textit{mv} command

\subsection{Learning Journal}
\subsubsection{New Techniques and Commands Learned:}
\begin{itemize}
    \learning{0mm}{Can be used to apply vspace without the command}
    \learning{rule}{Command can be used to rule a line accross the page}
    \learning{empty}{Value can be used to empty a default field (e.g. contentsname}
    \learning{asteriks}{At the end of a command (such as section) can be used to prevent it from including a number}
    \learning{hypperref}{Package can be used to link the table of contents to its location within the document}
    \learning{xcolor}{Package can be used to create custom colours for reference}
\end{itemize}

\subsubsection{Action Log}

\actionlog{29/08/2019 14:27}
\intention{Place a border around the title page in elaboration I}
\action{Attempted to use the \textit{frame} command around title page}
\result{Everything was removed except the data within a small frame}
\solution{Found code that helped me format my title page correctly \hyperlink{https://tex.stackexchange.com/questions/407812/page-border-for-cover-page-in-latex?rq=1}{here}} 

\actionlog{29/08/2019 14:58}
\intention{Remove the word contents from the Table of Contents command}
\action{Tried toc and table commands in the hopes that would remove it}
\result{Contents header was still printed}
\solution{I found another person with a similar issue and their solution was to use the \textit{def} command to define the contentsname variable to empty. Code found \hyperlink{https://latex.org/forum/viewtopic.php?t=8151}{here}.}

\actionlog{29/08/2019 18:04}
\intention{To remove the number before each section}
\action{Remove the begin section commands and replace them with just section* commands}
\result{The numbers were removed from the headers but unfortunately from the table of contents too}
\solution{Find a way to simply hide the numbers instead of removing them}

\actionlog{29/08/2019 18:11}
\intention{Hide the section numbers}
\action{Replace the section* commands with just section and introduce variable setcounter secnumdepth to 0}
\result{All numbers removed from all sections, not just Section}
\solution{Renew the section command only to prevent subsequent changes}

\actionlog{29/08/2019 18:30}
\intention{Add hyper reference links to the table of contents}
\action{Added the hyperref package and set the link colour to blue}
\solution{Worked correctly but links were too blue}

\actionlog{29/08/2019 23:30}
\intention{Create a new colour for referencing}
\action{Added package xcolor and used definecolor command using name,RGB,<R, G, B> then used newcolor name in hyperref parameters}
\result{A duller blue hyperlink}

\actionlog{29/08/2019 23:30}
\intention{Create a hyperlink to a URL}
\action{Used the \textit{hyperlink} command placing the label in the first argument and the link in the second}
\result{The label was the link and the URL was simply "here"}
\solution{Reverse the two arguments so clicking "here" takes you to the URL}
\newpage

%----------------------------------------------------------------Week 5
\newweek{}

\subsection{Data Carpentry (Unix Shell)}
\subsubsection{Pipes and Filters}
\question{What does \textnormal{>>} mean?}
The operator >> will print text as an addition to the files contents and not replace what is currently there.
\question{After the commands that correspond to the file animals-subset.txt:}
As the two commands use different operators the file will print the first three lines of animals.txt to the subset.txt - and then append it with the last two lines of animals.txt
\question{In order to find the 3 files which have the least number of lines in our directory - what command(s) would we need to run?}
wc -l * | sort-n | head -n 3
\question{What passes through each of the popes and the final redirect in the pipeline below?}
\begin{verbatim}
    cat animals.txt | head -n 5 | tail -n 3 | sort -r > final.txt
\end{verbatim}
This command will display the contents of the file animals.txt, print the first 5 lines, the next pipe will then filter to only print the last 3 lines of the 5 printed. Then the command will sort them in reverse order and save the results to the final.txt file
\question{Using uniq and another command, how can you list the animals without duplicating their names?}
The following command would achieve this
\begin{verbatim}
    cut -d , -f 2 animals.txt | sort | uniq
\end{verbatim}

\question{What command to produce a table that shows the total count of each type of animal in the file}
\begin{verbatim}
    cut -d , -f 2 animals.txt | sort | uniq -c
\end{verbatim}

\question{Can you match the same set of files with basic wildcard expressions?}
Yes just executing one command for A files and one for B
\question{Which of the following would remove only all the processed data files?}
\begin{verbatim}
    rm *.txt
\end{verbatim}

\subsubsection{Loops}
\question{Why do the commands have different outputs?}
One command will print the same thing for each file that matches the criteria. The other uses a variable allows it to loop and print the variable value once each

\question{What will be the output of the loop using c*?}
This will run the loop only on files beginning with c - and therefore will only displace cubane.pdb
\question{What is the effect of this loop? (Part 1)}
This will print the name and the contents into a file named alkanes. The last file processed will be the contents of alkanes.pdb (pentane.pdb)
\question{What is the output of this loop (Part 2)}
All the text from the .pdb files will be stored sequentially in the all.pdb file
\question{Dry Run Loop - do we want to run the command with weird quotation marks or not?}
We want to run Version 1 - version 2 will print the same thing for each file.\\
Edit: actually it appears the second version works - and will print the whole command within the command line
\question{Nested Loops: What would the following code do?}
The code will make 12 directories, each is named with two variables, the species and the temperature
\subsubsection{Shell Scripts}
\question{Write a script that takes any number of file names as command-line arguments}
I wrote a script 
\begin{verbatim}
    cut -d , -f 2 "$1" | sort | uniq
\end{verbatim}
But this didn't achieve all the files, i looked at the solution and should have added the loop and \$@ to ensure that it would repeat for all the different files present in the folder

\question{Why is history command listed in the history list?}
Because it was the last command that was executed. It can't run the history command without running it and logging it to history.

\question{Variable in Shell Scripts - What would i expect to see with a wildcard name and two arguments in a command}
In this instance i would expect to see the first and last line of each pdb file in the directory (if not an error from the quotation marks?)
\question{Write a script that takes the name of a directory and filename extension as its arguments}
To achieve this the script must use the wc command alongside two arguments and sorting to produce the correct result. The last two of the file are taken (as the last line is Total) and then you choose the head of the two.

\begin{verbatim}
    wc -l \$1/*.\$2 | sort -n | tail -n 2 | head -n 1
\end{verbatim}

\question{What would each script do?}
Script 1 - would print all the files in the current folder\\
Script 2 - would display the contents of 3 specified files \\
Script 3 - would print all pdb files that

\question{Debugging - what is wrong with the script?}
The variable is spelled incorrectly, should be \$datafile.

\subsubsection{Finding Things}
\question{Which command would result in the following output - and the presence of absence}

The command with the -w option, to prevent other words with 'of' letters in there
\question{Put these commands and pipes in the correct order}
\begin{verbatim}
    grep -w $1 -r $2 | cut -d : -f 2 | cut -d , 0f 1,3 > $1.txt
\end{verbatim}

\question{Using a for loop, how would you write a script that counts the number of each girls' name in the book?}
Script would need to perform 4 loops, one for each girls name through the same text which yields a single numbered result for each
\begin{verbatim}
    for girl in Jo Meg Beth Amy
    echo $girl
    grep -w $girl LittleWomen.txt | wc -l
    done
\end{verbatim}

\question{Matching and Subtracting - how to remove results with a second parameter?}
Easiest way is to filter the results again by using the v option.
\begin{verbatim}
    find data -name '*s.txt' | grep -v net
\end{verbatim}

\question{Write a short comment on the script}
It will return with all the .dat files and sort them by their number of lines 

\question{How to write a single search criteria for type, time of modification and user}
The script would need to match the files -type, modified within 1 day (mtime) and -user
\subsection{Learning Journal}

\subsubsection{New Techniques and Commands Learned:}

\paragraph{Overleaf}
\begin{itemize}
\learning{justify}{environment can be used instead of center to format the text evenly between margins}
\learning{href}{command should be used instead of \textit{hyperlink} to embody the link within text}
\learning{lmodern}{package can be used so symbols appear correctly instead of upside down question marks}
\end{itemize}
\paragraph{Unix Shell}
\begin{itemize}
\learning{wc}{command is used to count words, lines or characters of a file}
\learning{sort}{command can be used to order results}
\learning{head tail}{commands can be used to display the first and last records of a file}
\learning{cat}{used to display the contents of a file}
\learning{|}{can be used to write new commands that utilise the results of the previous}
\learning{variables}{in loops can be referenced with \$VAR}
\learning{for}{is the beginning of a loop, and must finish with done command}
\learning{quotation marks}{are used if values may have spaces in them}
\learning{\$@}{is used in scripts to execute on all potential files}
\learning{bash}{command is used to execute scripts}
\learning{grep}{command is used to search for words within files}
\learning{find}{command is used to find files}
\end{itemize}

\subsubsection{Action Log}
\actionlog{05/09/2019 12:56}
\intention{Used the hyperlink command in Overleaf to link text to a URL}
\action{Clicked the link}
\result{Took me to the table of contents}
\solution{Use the HREF command instead of hyperlink}

\actionlog{05/09/2019 13:47}
\intention{Format text as justify}
\action{Removed the center command and replaced it with justify}
\result{No formatting change took place}
\solution{Set the environment as justify instead of just a command, begin justify}

\actionlog{05/09/2019 22:30}
\intention{Write greater than symbols}
\action{Used these symbols as text}
\result{Displayed as upside down question marks}
\solution{Need to use package lmodern for symbols to be displayed correctly}


\actionlog{06/09/2019 00:20}
\intention{Write a script in Unix Shell}
\action{Wrote a loop script using the for command}
\result{Received error 'unexpected end of file'}
\solution{Edit the script again and add 'done' at the end}

\actionlog{06/09/2019 01:19}
\intention{Find a file in Unix Shell}
\action{Used command find . '*.txt'}
\result{Error: no such file or directory exists}
\solution{I needed to specify i was searching for the name, find . -name}

\actionlog{05/09/2019 23:53}
\intention{Write loop script for word count}
\action{Wrote script with looping and variable}
\result{Execution would just hang}
\solution{The variable was spelt incorrectly in the 3rd reference within the script}

\actionlog{05/09/2019 14:22}
\intention{Upload audio file for transcription}
\action{Uploaded my AAC file}
\result{oTranscriber says it's not supported by my browser}
\solution{Converted the file to a .wav and tried again}


\end{document}