\documentclass[12pt]{article}
 
 %-Packages
\usepackage[margin=1in]{geometry} 
\usepackage{titlesec}

%-Counters
\newcounter{problem} \setcounter{problem}{1}

%-Commands
\newcommand{\actionlog}[1]{\paragraph{Action Log ID: \theproblem\ -}{#1} ~\\ \addtocounter{problem}{1}}

\newcommand{\intention}[1]{\textbf{Intention:}{\textnormal\ #1} \newline}
\newcommand{\action}[1]{\textbf{Action:}{\textnormal\ #1} \newline}
\newcommand{\result}[1]{\textbf{Result:}{\textnormal\ #1} \newline}
\newcommand{\solution}[1]{\textbf{Solution:}{\textnormal\ #1} \newline}
\newcommand{\learning}[2]{\item \textit{#1} \textnormal{#2}}

%---------------------------------------------------------------
\begin{document}
%---------------------------------------------------------------

\begin{titlepage}
\title{Learning Journal}
\author{Aaron Hammond\\
FOAR705 - Digital Humanities}
\maketitle
\end{titlepage}



\begin{section}{FOAR705 Week 2}
\subsection{Data Carpentry (Spreadsheets)}

\subsubsection{Formatting data tables in Spreadsheets}
\linebreak

\textit{1.What are some common challenges with formatting data in spreadsheets and how can we avoid them?}\\
\\
The largest challenge with spreadsheets is formatting the information in a manner in which the information can be easily read regardless of who looks at it. The spreadsheet should be written that no prior knowledge is required to decode the information inside it. This means restricting the use of colours or any sort of coding device and instead extend the data table to include the additional information.\\
\\
Exercises:\\
\textit{1. Identify what is wrong with this spreadsheet (SAFI messy). Discuss the steps you would need to take to clean up the two tabs, and to put them all together in one spreadsheet}\\
\begin{enumerate}
    \item Rename SAFI\_messy to SAFI\_raw
    \item Create a new spreadsheet titled SAFI\_0.1
    \item In a single spreadsheet copy and paste the information from Mozambique 
    \item Label the headings of the columns as follows A. country B. key\_id C. type\_roof D. type\_floor E. no\_rooms F. includes\_barn G. oxen H. poultry I. goats J. cows K. total\_livestock L. no\_plots M. water\_use N. comments (for cowshed, look after cows, only in summer, dead cows info)
    \item Organise the data according to the information as possible, consolidating as many lines as needed and using Cut+Copy functions to ensure no doubling of data or missing of information
    \item Copy data from Tanzania into the same spreadsheet (delete empty tab)
    \item Populate the country field while organise the data according to the same principles used for Mozambique
    \item Delete any obviously incorrect data (e.g. -999) and replace with (Null) value to indicate the information collected for this field is not entirely credible
    \item Save changes
\end{enumerate}
\textit{2. Discuss this data with a partner and make a list of some of the types of metadata that should be recorded about this dataset}\\
\begin{enumerate}
    \item Who attended these interviews?
    \item What does the use of liv mean in the columns?
    \item How often is \textit{frequently ?}
    \item Does months\_lack\_food mean no meals or not a sufficient number of meals?
    \item instanceID is used in what type of software?
\end{enumerate}

\subsubsection{Find examples of problem in data produced by your discipline}
Within my discipline of Anthropology, locating problem data is very difficult. Typically ethnographies are written using field notes that are supported by academic literature. In saying this it is very difficult to locate problem sets of data as the public typically and privy to them.

\subsection{Learning Journal \date{15/08/2019}} 
\vspace{5mm}

\actionlog{No Timestamp}
\begin{itemize}
    \item Intention: Create a line break in LaTeX
    \item Action: Used the Rich Text tab and pressed enter
    \item Result: No line break
    \item Improvements or Solutions: Researched and found command "newline" to assist with formatting.
\end{itemize}

\actionlog{No Timestamp}
\begin{itemize}
    \item Intention: Create a custom title for each page
    \item Action: Used the 'title' command and 'maketitle' in different combinations
    \item Result: No title added
    \item Improvements or Solutions: Instead use the title page function for the initial title and utilise 'sections' instead to group areas of my work.
\end{itemize}

\actionlog{No Timestamp}
\begin{itemize}
    \item Intention: Create a numbered bullet list
    \item Action: Used 'itemize' command
    \item Result: Bulletted list without numbers
    \item Improvements or Solutions: Instead utilise the enumerate command for numbered lists
\end{itemize}

\actionlog{No Timestamp}
\begin{itemize}
    \item Intention: Use the underscore in naming conventions
    \item Action: Quoted a filename which utilised the \_ character
    \item Result: Massive issues in formatting and text running off the page
    \item Improvements or Solutions: When using underscores the backslash command must be used before the \_ symbol to ensure formatting remains intact
\end{itemize}

\actionlog{No Timestamp}
\begin{itemize}
    \item Intention: Create a counter so i don't have to manually count my problems
    \item Action: Created a counter and attempted to use stepcounter to print and increase it's value by 1 
    \item Result: Step counter did not print any characters
    \item Improvements or Solutions: Instead i used the counter command and referenced it using 'thecounter' and increased its value by 1 using 'addtocounter'afterwards
\end{itemize}

\actionlog{No Timestamp}
\begin{itemize}
    \item Intention: Fix the formatting issues caused by ending lists with vspace
    \item Action: Added vspace to the first Problem\_ID
    \item Result: All Problem\_IDs after the first are aligned differently
    \item Improvements or Solutions: Perhaps using a command other than vspace for formatting
\end{itemize}

\clearpage
\end{section}

\begin{section}{FOAR705 Week 3}

\subsection{Data Carpentry (Spreadsheets)}
\subsubsection{Date formats in spreadsheets}
\paragraph{Question: What year is shown in the year column?} ~\\
When adding data point '17/11' into the interview date column the Year column (formatted as =YEAR(A\$)) populates the field with '2019' (or the current year)
\newline
\paragraph{Key Points to Remember}
\begin{itemize}
    \item When working with dates within spreadsheets it is best to divide the three points of data into their own columns
    \item Do not allow the software to store the information as dates but integers instead to ensure they can be translated correctly to other software
    \item Excel will store dates as the current year unless specified otherwise
\end{itemize}

\subsubsection{Quality Assurance}

\paragraph{Key Points to Remember}
\begin{itemize}
    \item Data Validation tools help to mitigate errors and provide restrictions on the type of data that can be placed into columns
    \item Error messages can be customized as warnings or hard stops.
    \item Drop-down lists can be created per each data validation rule or can be sourced from within the document
    \item Overuse of these could cause issues if encountering unexpected  conditions or unique circumstances 
\end{itemize}

\subsubsection{Exporting Data}
\paragraph{Key Points to Remember}
\begin{itemize}
    \item File formats such as .xls and .xlsx are proprietary format and risk not being supported by future technology
    \item To ensure a document is more timeless and usable within a range of software, data should be saved as a .csv (comma seperated values) file type
    \item Within csv files ensure that any commas used in data fields are written within double-quotation marks to ensure that data is not reformatted incorrectly on re-opening
    \item Because it is formatted simply with commas and lines, .csv files can be edited in any text edit (such as notepad)
\end{itemize}

\subsection{Learning Journal}

\subsubsection{New Techniques and Commands Learned:}
\begin{itemize}
    \learning{subsubsection}{can be used to keep effective formatting within a subsection}
    \learning{tilde+doublebackslash}{can be used to force a line break after the paragraph command}
    \learning{verb}{can be used to ignore coding commands}
    \learning{newcommand[]}{can be used to create macros and within brackets you specify how many arguements you would like to include, arguments are referenced with the 'pound sign + number' combination}
    \learning{setcounter}{is used to set the value of a counter by specifying the counter name and the value within two separate curly bracket sets}
    \learning{backslash+space}{can be used to force a space}
    \learning{the+countername}{is the command to print the counter's value}
\end{itemize}

\subsubsection{Action Log}

\actionlog{22/08/2019 15:21}
\intention{Neatly format within a Subsection}
\action{attempted to use paragraph command to a standardized formatting}
\result{new line did not include referencing numbers for location}
\solution{Use the subsubsection command instead}

\actionlog{22/08/2019 15:36}

\end{section}

\end{document}